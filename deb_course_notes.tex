\documentclass[10,portrait]{article}
\usepackage{lmodern}
\usepackage{amssymb,amsmath}
\usepackage{ifxetex,ifluatex}
\usepackage{fixltx2e} % provides \textsubscript
\ifnum 0\ifxetex 1\fi\ifluatex 1\fi=0 % if pdftex
  \usepackage[T1]{fontenc}
  \usepackage[utf8]{inputenc}
\else % if luatex or xelatex
  \ifxetex
    \usepackage{mathspec}
  \else
    \usepackage{fontspec}
  \fi
  \defaultfontfeatures{Ligatures=TeX,Scale=MatchLowercase}
\fi
% use upquote if available, for straight quotes in verbatim environments
\IfFileExists{upquote.sty}{\usepackage{upquote}}{}
% use microtype if available
\IfFileExists{microtype.sty}{%
\usepackage[]{microtype}
\UseMicrotypeSet[protrusion]{basicmath} % disable protrusion for tt fonts
}{}
\PassOptionsToPackage{hyphens}{url} % url is loaded by hyperref
\usepackage[unicode=true]{hyperref}
\PassOptionsToPackage{usenames,dvipsnames}{color} % color is loaded by hyperref
\hypersetup{
            pdftitle={Dynamic Energy Budget (DEB) theory summary notes},
            colorlinks=true,
            linkcolor=pink,
            citecolor=red,
            urlcolor=blue,
            breaklinks=true}
\urlstyle{same}  % don't use monospace font for urls
\usepackage[margin=1in]{geometry}
\usepackage[]{biblatex}
\usepackage{longtable,booktabs}
% Fix footnotes in tables (requires footnote package)
\IfFileExists{footnote.sty}{\usepackage{footnote}\makesavenoteenv{long table}}{}
\IfFileExists{parskip.sty}{%
\usepackage{parskip}
}{% else
\setlength{\parindent}{0pt}
\setlength{\parskip}{6pt plus 2pt minus 1pt}
}
\setlength{\emergencystretch}{3em}  % prevent overfull lines
\providecommand{\tightlist}{%
  \setlength{\itemsep}{0pt}\setlength{\parskip}{0pt}}
\setcounter{secnumdepth}{0}
% Redefines (sub)paragraphs to behave more like sections
\ifx\paragraph\undefined\else
\let\oldparagraph\paragraph
\renewcommand{\paragraph}[1]{\oldparagraph{#1}\mbox{}}
\fi
\ifx\subparagraph\undefined\else
\let\oldsubparagraph\subparagraph
\renewcommand{\subparagraph}[1]{\oldsubparagraph{#1}\mbox{}}
\fi

% set default figure placement to htbp
\makeatletter
\def\fps@figure{htbp}
\makeatother


\title{Dynamic Energy Budget (DEB) theory summary notes}
\author{Matthew
Malishev\textsuperscript{1}*\\[2\baselineskip]\emph{\textsuperscript{1}
Department of Biology, Emory University, 1510 Clifton Road NE, Atlanta,
GA, USA, 30322}}
\date{}

\begin{document}
\maketitle

{
\hypersetup{linkcolor=black}
\setcounter{tocdepth}{4}
\tableofcontents
}
~

Date: 2018-11-05\\
R version: 3.5.0\\
*Corresponding author:
\href{mailto:matthew.malishev@gmail.com}{\nolinkurl{matthew.malishev@gmail.com}}

\newpage  

\subsection{List of parameters and
variables}\label{list-of-parameters-and-variables}

\begin{longtable}[]{@{}cl@{}}
\toprule
Parameter & Definition\tabularnewline
\midrule
\endhead
\(E\) & energy reserve\tabularnewline
\([E]\) & energy reserve per volume (reserve density)\tabularnewline
\(e\) & scaled energy reserve\tabularnewline
\([E_M]\) & maximum reserve density\tabularnewline
\(f\) & functional feeding response\tabularnewline
\(L\) & structural length\tabularnewline
\(\dot{p}_{AM}\) & maximum assimilation rate\tabularnewline
\(\dot{v}\) & energy conductance\tabularnewline
\bottomrule
\end{longtable}

\newpage  

\subsection{Overview}\label{overview}

Summarised notes from DEB workshops, telecourses, lectures, and
discussions.

\subsubsection{Reserve mobilisation}\label{reserve-mobilisation}

Conductance determines mobilisation rate from reserve to structure\\
The larger the surface area of reserve, the more mobilisation is
possible and thus faster maintenance and growth due to more surface
area.\\
- surface area scales slower than volume-specific energy flows.

Reserve dynamics f = 1 (max feeding rate)\\
\[
\frac
  {d E}
  {d t}  
  = \frac
  {f\{\dot{p}_{AM}\}}
  {L}  
  - \frac  
  {\dot{v}[E]}  
  {L}   
\]

\([E_{M}]\) = max reserve. Reserve doesn't change.

\[
= \frac
  {\{\dot{p}_{AM}\}}
  {L} - 
  \frac
  {\dot{v}[E_{M}]}
  {L}    
\] \[
\therefore 
[E_{M}]
= \frac
  {\{\dot{p}_{AM}\}}
  {\dot{v}}
\] Scaled reserve \[
e =
\frac
{[E]}
{[E_{M}]}  
\]

\[
\frac
{de}
{dt}
= \frac
{[E] /[E_{M}]}
{dt} 
= \frac
{f\dot{v}}
{L}  - 
\frac
{e\dot{v}}
{L}     
\]\\
\[
= \frac
{\dot{v} (f – e)}
{L} 
\]

Under steady state, reserve doesn't change\\
\[
0 =
\frac
{\dot{v} (f – e)}
{L}
\ \ \ \
\text{or}
\ \ \ \ f = e
\]

\subsubsection{Length}\label{length}

Getting maximum length \(L_{m}\)

\[
\frac
{dV}
{dt}
= V\dot{r}  
\]

Can rewrite \(r\) using scaled reserve \(e\)

\[
\dot{r} = 
\dot{v}
\frac
{\frac
{e}
{L} - 
(1 + \frac
{L_{T}} 
{L}
) /
{L_{m}}}
{e + g}
\] Getting \(L_{m}\)

\[
\frac
{dV}
{dt} = 
V\dot{r}
\]

To find \(V_{m} = Lm^{3}\), set \(f = 1\) and \(\frac{dV}{dt} = 0\),
then solve for \(V = V_{m}\)

\[
L_{m}
=
\frac
{\kappa\{\dot{p}_{Am}\}}
{[\dot{p}_{M}]}
\]

\subsubsection{Weak homeostasis}\label{weak-homeostasis}

Structural isomorphy implies weak homeostasis\\
Weak homeostasis depends on ratio of reserve to structure
\(\frac{d[E]}{dt}\)

\printbibliography

\end{document}
